%DO NOT MESS AROUND WITH THE CODE ON THIS PAGE UNLESS YOU %REALLY KNOW WHAT YOU ARE DOING
\begin{center}
\begin{huge}
\bfseries{Abstract}
\end{huge}
\end{center}

\noindent \setstretch{1.5} In recent years, biometric recognition and authentication has attracted a significant attention due to its potential applicability in social security, surveillance systems, forensics, law enforcement, and access control. A biometric system can be defined as a pattern-recognition system that can recognize individuals based on the characteristics of their physiology or behaviour. One biometric technique for unintrusive identification is gait recognition which basically identifies people by the way they walk. In former work, gait recognition is mainly achieved with camera systems. In this study, we present an approach for gait recognition using Microsoft Kinect V2, a peripheral for the gaming console XBOX One, which provides us with marker less tracking of human motion in real time. We extract and evaluate a number of static and kinematic features and present the results of various classification algorithms for person identification.
\pagebreak
